\section{Sobre o Curso}

O curso de Foresight da ESPM é um programa de pós-graduação lato sensu que visa capacitar profissionais para antecipar e moldar o futuro em diversos contextos organizacionais e sociais. Com uma abordagem interdisciplinar, o curso combina teoria e prática para desenvolver habilidades analíticas, estratégicas e criativas essenciais para o pensamento prospectivo.

Tradicionalmente, somos condicionados a pensar seguindo a sequência passado-presente-futuro. Este curso propõe uma mudança paradigmática: aprender a pensar o futuro de forma estruturada, utilizando métodos e ferramentas específicas que nos capacitem a compreender as dinâmicas de mudança e a construir cenários plausíveis. O objetivo é desenvolver a capacidade de abordar problemas complexos a partir dos futuros possíveis.

\section{Disciplinas}

\subsection{Futures Literacy: Introdução ao Foresight e aos Estudos do Futuro}

Esta disciplina introduz os fundamentos dos estudos do futuro, explorando como surgiu esta área de conhecimento e quais são seus conceitos fundamentais. Será apresentada a dinâmica de estruturação do campo, incluindo os principais contributors da UNESCO para o desenvolvimento da \textit{futures literacy}.

A disciplina apresentará o método CLA (\textit{Causal Layered Analysis}) de Inayatullah, uma ferramenta clássica e fundamental em \textit{futures studies}. O objetivo é proporcionar uma fundamentação teórica rigorosa e estruturante da área, oferecendo uma visão abrangente deste campo emergente e interdisciplinar.

\subsection{Teorias da Mudança}

Esta disciplina apresenta as principais teorias da mudança, capacitando os estudantes a identificar e analisar sinais fracos que podem indicar transformações futuras significativas. O foco está em desenvolver uma visão estrutural e de longo prazo para detectar mudanças emergentes, mesmo quando manifestadas em pequenos eventos ou tendências aparentemente isoladas.

Os estudantes aprenderão a reconhecer padrões e conexões entre movimentos aparentemente desconexos, compreendendo que muitas mudanças seguem direções específicas e possuem lógicas subjacentes que podem ser identificadas e analisadas sistematicamente.

\subsection{Fundamentos de Foresight: Escolas, Métodos e Práticas para Cenarização}

Esta disciplina concentra-se especificamente em Foresight, apresentando as principais escolas metodológicas da área. Serão exploradas principalmente duas abordagens distintas: a escola francesa, caracterizada por métodos mais quantitativos, estruturados e rigorosos (exemplificada pela abordagem Futuribles), e a escola inglesa, que privilegia métodos mais qualitativos, abertos e flexíveis (como a metodologia desenvolvida pela Shell).

O objetivo é proporcionar uma visão estrutural das diferentes formas de construir cenários úteis no contexto corporativo, desenvolvendo sensibilidades para a gestão de riscos e a aplicação prática de métodos de cenarização.

\subsection{Análise de Tendências e Pesquisa de Mercado}

Esta disciplina foca no desenvolvimento de competências para análise sistemática de tendências e elaboração de relatórios prospectivos. Os estudantes aprenderão metodologias específicas para identificar, classificar e interpretar tendências emergentes, bem como técnicas para comunicar insights de forma clara e estruturada em relatórios profissionais.

\subsection{Estratégia e Criatividade: Como Gerar Insights a partir do Pensamento Complexo e Sistêmico}

Esta disciplina desenvolve a capacidade de pensar o futuro dentro de uma perspectiva sistêmica e complexa. Os estudantes aprenderão a identificar e mapear as interconexões e consequências entre diferentes esferas da existência humana e organizacional.

\begin{thinkerquote}
\textit{"Eu posso plantar uma batata e, dependendo da sociedade, isso pode ser um evento religioso."}
\end{thinkerquote}

O foco está em compreender as conexões dinâmicas entre economia, sociedade, tecnologia, política, meio ambiente, cultura, aspectos globais, psicológicos, éticos, legais e demográficos, desenvolvendo a habilidade de desenrolar o fio condutor de consequências em sistemas complexos.

\subsection{Pesquisa Qualitativa em Futuros: Questionamento Crítico como Ferramenta}

Esta disciplina tem o papel fundamental de ensinar a formulação de questões apropriadas para o pensamento prospectivo. Uma vez que somos tradicionalmente programados para pensar na sequência passado-presente-futuro, este curso propõe uma desconstrução mental para abordar problemas a partir dos futuros possíveis.

Os estudantes desenvolverão uma metodologia específica para interrogar objetos de estudo de forma diferenciada, aprendendo a se apropriar de situações e problemas, formular questões pertinentes e estruturar investigações prospectivas. Esta disciplina marca o início da construção de uma caixa de ferramentas metodológicas com enfoque crítico.

\begin{keypoint}
As seis disciplinas anteriores constituem o bloco fundamental do programa, proporcionando a ambientação necessária e a exploração inicial de ferramentas de foresight. Este conjunto de conhecimentos é essencial para o desenvolvimento das competências avançadas de construção de matrizes de incertezas e elaboração de cenários.
\end{keypoint}

\subsection{Mapeamento de Incertezas}

Esta disciplina introduz o conceito e as metodologias para trabalhar com incertezas de forma sistemática. Os estudantes aprenderão a identificar, mapear e gerenciar incertezas, desenvolvendo competências para construir matrizes e mapas de incertezas como base para a elaboração de cenários.

O foco está em retomar e aplicar os sinais identificados em disciplinas anteriores, hierarquizando elementos segundo critérios de incerteza, probabilidade, criticidade e impacto. Esta abordagem estruturada permite uma análise mais precisa dos fatores que influenciam a construção de cenários futuros.

\subsection{Megatendências}

Esta disciplina explora o conceito de megatendências como forças motrizes globais de longo prazo que impactam a sociedade, a economia, a tecnologia e o meio ambiente. Exemplos incluem urbanização, digitalização, sustentabilidade, envelhecimento populacional e globalização.

Os estudantes desenvolverão competências para identificar, analisar e interpretar megatendências, compreendendo como essas forças podem influenciar o futuro das organizações e da sociedade. A disciplina proporciona ferramentas para incorporar a análise de megatendências nos processos de planejamento estratégico e construção de cenários.


\subsection{Foresight Estratégico: Construção de Cenários Corporativos}

Esta disciplina aprofunda a aplicação da escola francesa de Foresight, utilizando métodos estruturados e quantitativos para construção de cenários corporativos. Será explorada a metodologia de Michel Godet e ferramentas práticas para hierarquização de incertezas em contextos organizacionais.

Os estudantes aprenderão a desenvolver cenários robustos utilizando planilhas e métodos sistemáticos, adequados para aplicação em ambiente corporativo e tomada de decisões estratégicas.

\subsection{Treinamento Prático de Foresight: Os Bastidores de Cases Globais}

Esta disciplina examina casos reais de aplicação de Foresight em contextos globais, com foco na análise crítica dos fatores que contribuem para o sucesso ou fracasso de projetos prospectivos. 

Os estudantes desenvolverão sensibilidades práticas para identificar armadilhas comuns em trabalhos de Foresight, aprendendo a reconhecer elementos problemáticos e a criar estratégias para elaboração de relatórios de alta qualidade. A disciplina proporciona uma perspectiva realista sobre os desafios práticos da profissão de futurista.

% Disciplinas complementares em desenvolvimento:
% \subsection{Design Especulativo e Futures Thinking Aplicado à Construção de Cenários}
% \subsection{Foresight e Sustentabilidade: Projetando Cenários para Cidades Inteligentes}